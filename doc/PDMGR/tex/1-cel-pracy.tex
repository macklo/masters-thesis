\newpage % Rozdziały zaczynamy od nowej strony.
\section{Cel pracy}
Celem pracy jest zbadanie możliwości wykorzystania logiki rozmytej do procesu modelowania oraz do sterowania obiektami nieliniowymi. Na początek badana będzie jedna klasa obiektów nieliniowych - reaktory chemiczne. Celem jest zaproponowanie podejścia do procesu modelowania, który sprawdzałby się dla różnych obiektów tej kategorii. Uzyskany model, wykorzystujący logikę rozmytą, powinien dawać lepsze rezultaty pod względem wydajności, dokładności oraz prostoty zastosowania. Uzyskany model zostanie wykorzystany do zaprojektowania rozmytego regulatora nieliniowego obiektu.

